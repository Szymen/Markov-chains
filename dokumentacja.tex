\documentclass{article}
\usepackage[utf8]{inputenc}
\setlength{\parindent}{1cm}

\title{Projekt JiMP 2\\\textbf{Automatyczna generacja tekstu}}
\author{Szymon Maslowski }
\date{Semestr letni 2015}
\begin{document}

\maketitle

\section*{Specyfikacja funkcjonalna programu Markov}

\subsection*{Nazwa}

\textbf{markov} - program generujący tekst losowy na podstawie zadanego tekstu/tekstów -- interfejs linii poleceń\\


\textit{Przykładowe użycie} :

\$ markow 2 tekst1.txt text2.txt 130 wynik.out 

\$ markow test2 200 4 -S statystyki.out
	

\subsection*{Streszczenie}
\par markov [-d](opcjonalne) nazwa\_pliku\_z\_bazą\_ przejsc n \{nazwy plików wejsciowych\} -długość\_tekstu\_wyjściowego  -(opcjonalna)ilość\_akapitów -plik\ wynikowy  -(opcjonalna)długość\_n-gramu  [-S]{-S statystyki}-(opcjonalna)nazwa\_pliku [-c](opcjonalne) nazwa\_pliku\_do\_zapisu\_bazy\_przejść
	
\subsection*{Opis}
\par Markov to program generujący, na podstawie analizy tesktów, losowe ciągi słów w języku naturalnym. \\
		
		\textbf{-n}\\
		
			Ilość plików wejściowych zawierających tekst, domyślnie 1.\\
		
		\textbf{-d  nazwa\_pliku\_z\_bazą\_przejsc}\\
		
		    Możliwe jest wywołanie programu podając gotową już bazę przejść, stworzoną przy poprzednich wywołaniach programu.\\
		
		\textbf{-nazwy plików wejściowych}\\
		
		    N plików wejściowych binarnego tekstu, zawierających tekst w języku naturalnym, na podstawie, których zostanie wygenerowany tekst wyjściowy\\
		
		\textbf{-długość tekstu wyjściowego}\\
		
			Docelowa długość wygenerowanego tekstu. Margines błędu ~5 \%\\
		
		\textbf{-ilość akapitów}\\
		
			Ilość akapitów, które ma mieć tekst, objętość akapitów będzie taka sama z niepewnością ~5\%. Domyślnie 1 akapit\\
		
		\textbf{-plik wynikowy} \\
		
			Nazwa pliku, do którego zostanie zapisany efekt działania programu. Jeżeli nie podano program wypisze wynik do konsoli\\
		
		\textbf{-długość n-gramu}\\
		
			Długość n-gramu na podstawie, którego budowane są zależności w słowach tekstu. Domyślnie 3\\
			
		\textbf{-S statystyki}\\
		
			Jeżeli wybrano program z tym parametrem, to oprócz normalnego wygenerowania tekstu losowego, program wypisze też do zadanego pliku(jeżeli nie podano to wypisze do konsoli) statystyki dotyczące najczęściej wykorzystywanych słów, najczęsciej powtarzających się n-gramów, PMI(różnica między oczekiwaną częstotliwością występowania ciągów słów, a oczekiwaną) \\
			
			\textbf{-c nazwa\_pliku\_do\_zapisu\_bazy\_przejść}\\
			
			Możliwe jest zachowanie bazy możliwych przejść w pliku o podanej nazwie. Zawartość zostanie nadpisana! W celu tylko zwiększenia istniejącej bazy należy wywołać program podając teksty wejściowe i bazę danych,tekst wyjściowy długości 0 i wybrać ten parametr. 
			

\subsection*{Błędy w użyciu}

W razie wystąpienia błędów, komunikat zostanie wypisany do pliku "markov\_errs" i na strumień błędów. Program stara się wygenerować tekst bez błędów, dlatego w razie wystąpienia nie właściwych parametrów wejściowych, stosowna informacja zostanie wydrukowana na standardowy strumień błędów a program spróbuje działać dalej przyjmując wartości domyślne.\\
Np.:

\$ markow 2 130 wynik.out\\
makrov:  blad! nie podano plikow wejsciowych!\\

\$ markow 2 tekst1.in tekst2.in -130 wynik.out\\
markov: blad! ilosc tekstu nie moze byc ujemna!\\

\subsection*{Inne}
Kontakt : szymmaslowski@gmail.com\\
\begin{center}
Wykonane jako praca zaliczeniowa na projekt z JiMP2.\\Politechnika Warszawska 2015.
\end{center}


\end{document}
